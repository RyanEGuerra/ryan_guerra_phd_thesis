\chapter{WURC Console API}
\label{sec_wurc_console}

The following commands summarize the high-level \ac{API} provided by the \ac{WURC} on-board firmware via its console interface.
Each command is a single character followed by an optional operand. A command is stored in a character buffer where it can be edited or deleted until the ENTER keystroke is issued, at which case the console interpreter will attempt to process the command.

Special characters [0, 9] represent calibration tuning commands and are execute instantly without waiting for an ENTER keystoke. 

The host can read back the output console stream to monitor command execution; the string \texttt{wss\$} is printed when a command is finished executing.

\section{WURC Macro Console API}

\begin{singlespace}
\small
\begin{verbatim}
============== WURC Firmware Available Commands ===================
h : print this help menu
c <###> : display or edit RF calibration tables
c : Displaying the Calibration Table
c <#GGXXYYD> : Edit the Volatile LOFT Calibration
c <#FFAAAAAAAABBBBBBBBCCCCCCCCD> : Edit the Volatile IQ Calibration
s : commit volatile RF calibration tables to non-volatile memory
A : run transceiver initialization sequence
F : run core LMS DCO calibration sequence (development only)
C <T##BB> : set channel macro
x : toggle baseband control of Tx/Rx switching (default = DISABLED)
i : get WSD serial number (DEPRECATED)
j : get transceiver info, printed as a JSON string
r <HH> : read LMS register HH
w <HHKK> : write KK to LMS register HH
l : toggle WURC indicator LED
e : clear WURC error indicator LED
u : enable/disable automatic calibration value loading (dev only)
B <kHz> : set receive center frequency
D <kHz> : set transmit center frequency
R : toggle LMS6002D soft receive enable
S : toggle LMS6002D soft transmit enable
X : reset the transceiver and disable all transmit power amplifiers
U : set the transceiver into DAC ADC mode (buggy)
T: toggle RXOUTSW state
P : print current LOFT DCO values
E : enable RF loopback mode (development)
I <T, R, t, r> : set the digital DAC/ADC I/Q polarity or I/Q intlv
d : perform a configuration register dump (development)
0, 1, 2, 3, 6, 7, 8, 9 : increment or decrement Tx/Rx I/Q LOFT DCO
N : set WURC as transmitter on ANT1 (BB ctrl x overrides this)
M : set WSD as receiver on ANT1 (BB ctrl x overrides this)
Z : set transceiver mode (development)
V <0:15> : set transmit baseband LPF bandwidth
W <0:15> : set receive baseband LPF bandwidth
y <0/1><#> : set the PLL delay count (development)
z <0/1> : set Tx Rx switching mode (development)
n, G, H : Transmit Gain Settings
n <0:56> : set log-linear transmit gains
G <0:31> : set transmit TXVGA1 gain (development)
H <0:25> : set transmit TXVGA2 gain (development)
J, K, L : Receive Gain Settings
J <0:120> : set receive RXVGA1 gain (linear)
K <0:10> : set receive RXVGA2 gain (log-linear)
L <0:2> : set receive RXLNA gain (log-linear)
=================================================================
\end{verbatim}
\end{singlespace}


%\section{WARP MicroBlaze Console API}
%
%\begin{singlespace}
%\small
%\begin{verbatim}
%
%TODO
%
%\end{verbatim}
%\end{singlespace}